\documentstyle[named,mymacros,epsf,11pt]{article}
\begin{document}

\newcommand{\mona}{{\sc morphix++}}
\newcommand{\morphix}{{\sc morphix}}

\title{A Fast and Portable Morphological Component 
        for Inflectional Languages}


\author{G\"{u}nter~Neumann}

\date{Deutsches Forschungszentrum f\"ur K\"unstliche 
Intelligenz GmbH\\
Stuhlsatzenhausweg 3\\
66123 Saarbr\"ucken, Germany\\
{\tt neumann@dfki.uni-sb.de}}


\maketitle

\section{Introduction}

In this technical paper we describe \mona\ the morphological
component developed at the language technology lab of the DFKI
GmbH. \mona\ is a descendant of \morphix\, a fast
classification-based morphology  component for German
\cite{Morphixmemo:86,Morphix:88}. \mona\ improves \morphix\ in that the
classification-based approach has been combined with the well-known
two-level approach, originally developed by
\cite{Koskenniemi83,Koskenniemi84}. Actually, the extensions concern

\begin{itemize}

\item   the use of tries as the unique storage device for all sort of
        lexical information in \mona\
        (e.g., for lexical entries, prefix, inflectional endings);
        besides the usual functionality (insertion, retrieval,
        deletion), a number of more complex functions are available 
        most notably a regular trie matcher, and

\item   online processing of compounds which is realized 
        by means a recursive trie traversal. 
        During traversal two-level rules
        are applied for recognizing possible decompositions
        of the word form in question. \mona\ also supports
        robust processing of compounds, such that
        compounds are recognized by means of longest
        matching substrings found in the lexicon; e.g., 
        the word ``adfadfeimer''
        will return result for ``eimer'' assuming that ``adfadf'' is no legal
        lexical stem.

\item   A generic and parameterizable output interface for \mona
        has been implemented which returns a
        normalized feature vector representation of the computed
        morpho-syntactic information. It is possible to parameterize
        the set of relevant morpho-syntactic features which supports lexical
        tagging and feature relaxation.

\item   On basis of this new interface a  very efficient specialized
        constraint solver (called {\sc munify}) has been implemented 
        which can be used in combination with parsing techniques
        in order to perform unification of the morpho-syntactic
        information.
              

\end{itemize}

Thus, the basic processing strategy employed by \mona\ consists of 
trie traversal combined with the application of finite state automata.

Currently, \mona\ has been used for the German and Italian language.
An English version is under way. \mona\ is implemented in Lisp
following ansi Allegro Common Lisp. It has been tested under Solaris,
Linux, Windows~98 and Windows-NT.
The German version has a very broad coverage (a lexicon of
more then 120.000 stem entries), and an excellent speed (5000 words/sec
without compound handling, 2800 words/sec with compound processing
(where for each compound all lexically possible
decompositions are computed)\footnote{
Measurement has been performed on a Sun~20 using an on-line lexicon of
120.000 entries.}

\section{The main control flow and toplevel function}



\section{Lexicon Management}

\subsection{The Trie abstract data type}
The lexicon is implemented using a Trie abstract data type.
The Trie implementation comes with its own package named ``TRIE''. The
toplevel functions are:

\begin{itemize}
  
\item   MAKE-TRIE (\&OPTIONAL (NAME *TRIE*)):\\
        creates a trie; if name is specified, then the new trie is bound
        to name; otherwise the  default name *trie* is used
        Example: (make-trie *my-trie*) returns a trie bound  to *my-trie*

\item   INSERT-ENTRY (KEY INFO LEXICON \&KEY (OVERWRITE NIL) 
                                             (LEAF-NAME :INFO)):\\

        inserts INFO under path KEY in LEXICON using the leaf-name
        LEAF-NAME; if OVERWRITE is T, then the complete old INFO is 
        overwritten

\item   GET-ENTRY (KEY LEXICON \&KEY (LEAF-NAME :INFO)):\\
        traverses LEXICON using KEY and returns info located under LEAF-NAME,
        if present, otherwise returns nil
  
\item   DELETE-ENTRY (KEY LEXICON
		     \&KEY (LEAF-NAME :INFO):\\
        deletes info found under path KEY in LEXICON located under LEAF-NAME
        (note, this function not only deletes the information found under KEY
        but also deletes the path, if possible, i.e., reorganizes the tree;

\item   ENTRY-EXISTS-P (KEY LEXICON \&KEY (LEAF-NAME ':INFO)):\\
        a boolean function which informs you whether lexicon contains 
        information reachable by path KEY located under LEAF-NAME


\item   my implementation uses sequence as a general date type;
        thus it is possible to use either strings, vectors, or lists as 
        the data structures for the TRIE; the current default is
        strings

\item   regular Trie matcher:

\end{itemize}

\subsection{Toplevel functions for the lexicon}

There are two tries used,
*stem-lexicon*, *fullform-lexicon*; the functions used to operate on these
tries are defined in the morphix package ``MORPHIX'':
\begin{itemize}

\item   insertion:\\
        (l-i stem info lexicon), 
        where stem should be a string and lexicon one
        of the lexicons *stem-lexicon* and *fullform-lexicon*. INFO is
        the integer representation of the corresponding POS-specific
        information.

\item   retrieval:\\
        (l-z stem lexicon)

\item   deletion:\\
        (l-r stem lexicon)

\end{itemize}


\section{Processing of compounds}

\section{Generation with \mona}

\section{Flexible user interface}

\subsection{Morphological features}
\label{morph-bnf}

The following table enumerates the features together with
their domain:

\begin{tabular}{lll}
:cat     & $\rightarrow$ & :n :v :aux :modv :a :attr-a :def :indef :prep\\
         &               & :relpron :perspron :refpron :posspron :whpron \\
         &               & :ord :card :vpref :adv :whadv \\
         &               & :coord :subord  :intp :part\\
:mcat    & $\rightarrow$ & :n-adj :det-word\\
:sym     & $\rightarrow$ & :open-para :closed-para :!sign :questsign\\ 
         &               & :seperator :dot :dotdot :semikolon :comma\\
:comp    & $\rightarrow$ & :p :c :s\\
:comp-f  & $\rightarrow$ & :pred-used\\
:det     & $\rightarrow$ & :none :indef :def\\
:tense   & $\rightarrow$ & :pres :past :subjunct-1 :sibjunct-2\\
:form    & $\rightarrow$ & :fin :infin :infin-zu :psp :prp \\
         &               & :prp-zu :imp \\
:person  & $\rightarrow$ & 1 2 :2a 3 :anrede\\
:gender  & $\rightarrow$ & :nfm :m :f :nt\\
:number  & $\rightarrow$ & :s :p\\
:case    & $\rightarrow$ & :nom :gen :dat :akk\\     
\end{tabular}

\section{A small user guide}







\bibliographystyle{named}
\bibliography{bibtex-abbrev,morpholo,neummy}
\end{document}